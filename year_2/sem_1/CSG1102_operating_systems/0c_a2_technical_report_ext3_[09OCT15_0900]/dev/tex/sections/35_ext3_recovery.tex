\subsection{Recovery in ext3}

Until a system failure occurs, the contents of the journal are not used \citep{Bovet2006}. However, when the filesystem is uncleanly unmounted during a disk operation, the next mount of the filesystem will highlight transactions in the journal that have yet to be checkpointed to their fixed locations on disk \citep{Katiyar2011}. In such cases, \mintinline{console}{e2fsck} is triggered, scanning the journal and replaying completed transactions required to make the filesystem consistent again \citep{Bovet2006}. \citet[p. 8-9]{Katiyar2011} describes the recovery process in stepwise fashion:

\begin{enumerate}
\item \mintinline{console}{journal_recover()} called
	\begin{itemize}
	\item Readahead journal blocks in memory
	\end{itemize}
\item \mintinline{console}{do_one_pass(PASS_SCAN)} called
	\begin{itemize}
	\item This is the first scan of the journal, determines if recovery is required
	\item If recovery is required, sanity checks are performed to determine which transactions need to be replayed and confirms if the journal is valid
		\begin{itemize}
		\item Only completed transactions are considered, incomplete transactions are discarded \citep{Bovet2006}
		\item Data structure \mintinline{console}{recovery_info} is populated with information required for the recovery
			\begin{itemize}
			\item Such as \mintinline{console}{s_start}, defining which block number to begin recovery
			\end{itemize}
		\end{itemize}
	\item If recovery is not required, \mintinline{console}{s_start = 0}
	\end{itemize}
\item \mintinline{console}{do_one_pass(PASS_REVOKE)} called
	\begin{itemize}
	\item Builds a list of revoked blocks which is referred to during recovery process, ensuring that old data in a block is not overwritten by new data which would cause corruption
	\end{itemize}
\item \mintinline{console}{do_one_pass(PASS_REPLAY)} called
	\begin{itemize}
	\item Replays transactions from the journal and copies data from the journal to fixed locations on disk
		\begin{itemize}
			\item Read the corresponding block number from the filesystem
			\item Copy contents of journal to buffer
			\item Mark the buffer dirty
			\item Buffer written to fixed location on disk
		\end{itemize}
	\end{itemize}
\item \mintinline{console}{journal_clear_revoke()} called
	\begin{itemize}
		\item List of revoked blocks destroyed once replay is completed, space can be reclaimed
	\end{itemize}
\item \mintinline{console}{sync_blockdev()} called
	\begin{itemize}
	\item Sync block device
	\end{itemize}
\item \mintinline{console}{journal_reset()} called
	\begin{itemize}
	\item Resets in-memory fields of the journal, allowing it to operate as normal again when the filesystem is remounted
	\end{itemize}
\end{enumerate}