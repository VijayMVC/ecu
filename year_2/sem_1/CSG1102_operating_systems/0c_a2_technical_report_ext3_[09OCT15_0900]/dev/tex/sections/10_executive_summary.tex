\section{Executive summary}

This report provides a description of the implementation and operation of the ext3 filesystem. The introduction of journaled filesystems describes the advantages of a journaled filesystem compared to traditional, non-journaled filesystems, providing greater availability of the system in the event of an improper shutdown or crash.

Ext3's inter-relation and similarities with ext2 is examined to provide context for the reasons behind ext3's development and design decisions. Compatibility with the existing non-journaled ext2 filesystem was found to be a key design element to ensure that the majority of Linux users would experience a seamless transition to the new ext3 filesystem, while providing journal functionality.

The implementation of ext3 is discussed, exploring the concepts that enable journaling in the ext3 filesystem. A separate layer, called the \emph{Journaling Block Device} (JBD), implements the journaling capabilities in ext3, and is encapsulated from the filesystem itself. The decision to separate the concern of journaling stems from the design goal of maintaining compatibility with ext2, and allows the ability for the JBD to be used with any other journaled filesystem.

The implementation of journaling in ext3 is further described by examining the journaling approach, the contents of the journal itself, and the logical structure of the journal. The three modes of ext3 operation are also discussed which provide the user a choice of journal verbosity, depending on the user's desire for filesystem performance and data consistency guarantee. Additionally, the process of checkpointing is discussed, a function that verifies that each completed transaction has its fixed location, on disk transfers have occurred.

Finally, descriptions of ext3 operations are discussed to provide an insight into how the ext3 filesystem functions during a write system call, as well as the recovery of an uncleanly unmounted ext3 filesystem.