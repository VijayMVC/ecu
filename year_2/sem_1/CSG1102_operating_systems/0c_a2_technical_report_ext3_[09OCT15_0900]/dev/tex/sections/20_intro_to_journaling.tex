\section{Introduction to journaled filesystems}

One of the most important parts of an operating system is the filesystem. It contains and manages critical data on disk drives, such as user configuration, user data and applications and of course, the operating system itself. The filesystem ensures that what is read from storage is identical to what was originally written. In other words, it ensures data integrity. A filesystem achieves data integrity by managing and storing not only the actual data, but also information about the data, or files in storage. It also stores and manages information about the filesystem itself. This information is known as metadata \citep{Best2002}.

But what happens to data when a filesystem is uncleanly unmounted during file operations due to a system crash or abrupt power loss? When Linux detects an uncleanly unmounted filesystem, it signals a non-journaled filesystem (such as ext2) to perform a consistency check with \mintinline{console}{fsck} before booting into the operating system. This function scans the entire filesystem and attempts to fix any detected consistency issues that can be safely resolved. The \mintinline{console}{fsck} utility may take a considerable amount of time, depending on size of the filesystem, which can equate to serious downtime for the user \citep{Jones2008}. According to the developer of the ext3 filesystem, \citet{Tweedie2000}, Linux filesystems are growing in size, causing longer \mintinline{console}{fsck} process times.

Journaled filesystems negate the need for \mintinline{console}{fsck} (or equivalent in non *nix operating systems) to verify filesystem integrity, therefore greatly reducing downtime and increasing system availability in the event of an unclean unmount \citep{Best2002, Bost2012, Jones2008, Prabhakaran2005a, Tweedie1998, Tweedie2000}. This is achieved by implementing a \emph{log} or \emph{journal} that records changes destined for the filesystem. The changes recorded in the journal are committed to the filesystem periodically, and when an unclean unmount occurs, the journal is referenced as a checkpoint to recover unsaved information to avoid corrupted filesystem metadata \citep{Jones2008}. Therefore, a \mintinline{console}{fsck} scan of the entire filesystem is no longer required to determine consistency after an unclean unmount, as the journal is used for reference to determine integrity in a far quicker manner.