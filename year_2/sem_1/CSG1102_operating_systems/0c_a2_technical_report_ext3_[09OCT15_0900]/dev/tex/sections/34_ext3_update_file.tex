\subsection{Writing data block of regular file in ext3}

\citet[p. 772-774]{Bovet2006} describe the processes involved when the ext3 filesystem is requested to write data blocks of a regular file in a stepwise fashion.

\begin{enumerate}
	\item A call to system function \mintinline{console}{write()} triggers \mintinline{console}{generic_file_write()} for the file object associated with the ext3 file
	\item \mintinline{console}{ext3_prepare_write()} is called for the \mintinline{console}{address_space} object for each page of data relating to the write operation
	\item A new atomic operation is initiated from \mintinline{console}{ext3_prepare_write()} by calling the JBD API function \mintinline{console}{journal_start()}, creating a new handle
		\begin{itemize}
			\item First call to \mintinline{console}{journal_start()} will result in newly created handle to be added to active transaction
			\item Subsequent calls to \mintinline{console}{journal_start()} for the same handle verifies that the \mintinline{console}{journal_info} field of the process descriptor is set and uses the referenced handle
		\end{itemize}
	\item \mintinline{console}{ext3_prepare_write()} calls \mintinline{console}{block_prepare_write()}, passing the address of \mintinline{console}{ext3_get_block()} function
		\begin{itemize}
		\item \mintinline{console}{block_prepare_write()} prepares the buffers and buffer heads of the file's page
		\end{itemize}
	\item Kernel calls \mintinline{console}{ext3_get_block()} to determine the logical number of a block from the ext3 filesystem
		\begin{itemize}
		\item \mintinline{console}{journal_get_write_access()} is called before issuing a low level write operation on a metadata block of the filesystem, to add the metadata buffer to a list of the active transaction
			\begin{itemize}
			\item If the metadata is already included in an existing and incomplete transaction, the buffer is duplicated, ensuring that older transactions are committed with the old content
			\end{itemize}
		\item \mintinline{console}{journal_dirty_metadata()} is called after updating the buffer containing the metadata block, moving the metadata buffer to the dirty list of the active transaction and logging the operation in the journal
		\end{itemize}
	\item If using data journal mode, \mintinline{console}{ext3_prepare_write()} calls \\ \mintinline{console}{journal_get_write_access()} on every buffer affected by the write operation, recalling that in data journal mode, metadata as well as data is recorded to the journal
	\item \mintinline{console}{generic_file_write()} receives control again, updating the page page with the data stored in the User Mode address space, then calls \mintinline{console}{commit_write()} for the \mintinline{console}{address_space} object, depending on the ext3 mode:
		\begin{itemize}
		\item In data journal mode, \mintinline{console}{commit_write()} is implemented by \\ \mintinline{console}{ext3_journalled_commit_write()}, calling \mintinline{console}{journal_dirty_metadata()} on every buffer of data in the page
			\begin{itemize}
			\item This ensures that data is also included in the proper dirty list of the active transaction and not in the dirty list of the owner inode
			\item Also ensures that the corresponding log records are written to the journal
			\item Again, recall that data journal mode writes both data and metadata to the journal
			\item \mintinline{console}{ext3_journalled_commit_write()} then calls \mintinline{console}{journal_stop()} so that the JBD closes the handle for that particular operation
			\end{itemize}
		\item In ordered mode, \mintinline{console}{commit_write()} is implemented by \\ \mintinline{console}{ext3_ordered_commit_write()}, calling \mintinline{console}{journal_dirty_data()} on every buffer of data in the page
			\begin{itemize}
			\item This inserts the buffer in a proper list of the active transactions
			\item The JBD then ensures that all data buffers in this list are written to disk before the metadata buffers of the transaction
			\item The JBD does not include log records to the journal at this stage in ordered mode mode
			\item \mintinline{console}{ext3_ordered_commit_write()} then calls \\ \mintinline{console}{generic_commit_write()}, to insert the data buffers in the list of the dirty buffers of the owner inode
			\item \mintinline{console}{ext3_ordered_commit_write()} then calls \mintinline{console}{journal_stop()} so that the JBD closes the handle for that particular operation
			\end{itemize}
		\item In writeback mode, \mintinline{console}{commit_write()} is implemented by \\ \mintinline{console}{ext3_writeback_commit_write()}, calling \mintinline{console}{generic_commit_write()}
			\begin{itemize}
			\item This inserts the data buffers in the list of the dirty buffers of the owner inode
			\item \mintinline{console}{ext3_writeback_commit_write()} then calls \mintinline{console}{journal_stop()} so that the JBD closes the handle for that particular operation
			\end{itemize}
		\end{itemize}
	\item \mintinline{console}{write()} system call finishes here, but the JBD continues processing
		\begin{itemize}
		\item When all log records have been physically written to the journal, the transaction status will be changed to complete
		\item When complete, \mintinline{console}{journal_commit_transaction()} is called
		\end{itemize}
	\item While in ordered mode, \mintinline{console}{journal_commit_transaction()} ensures I/O data transfers for all data buffers included in the list of the transaction first, and waits until all data transfers are completed
	\item While in writeback or data journal mode, \mintinline{console}{journal_commit_transaction()} initiates I/O data transfers for associated metadata buffers for the transaction
		\begin{itemize}
		\item Additionally for data journal mode, I/O data transfers are also initiated for data buffers
		\end{itemize}
	\item The kernel periodically initiates checkpointing of all complete transactions residing in the journal, verifying that I/O transfers have successfully completed
		\begin{itemize}
		\item If so, the transaction is removed from the journal to make room for new transactions
		\end{itemize}
\end{enumerate}