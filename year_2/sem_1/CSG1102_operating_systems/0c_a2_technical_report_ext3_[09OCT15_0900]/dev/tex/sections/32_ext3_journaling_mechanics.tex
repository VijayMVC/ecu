\section{Ext3 journal mechanics}

With backwards/fowards compatibility in mind, journaling functionality has been added to ext2 to create the ext3 filesystem by implementing two main concepts, the Journaling Block Device, and transactions.

The ext3 journal itself is stored in an inode using a circular buffer data structure \citep{Robbins2001a, Tweedie2000, Jones2008, Prabhakaran2005a}. An inode is an area on the disk where all the information about a file is stored, however, the actual file itself is not stored in an inode. \citet[p. 2]{Best2002} uses the analogy of a ``bookkeeping file for a file'', and asserts that an inode is in fact a file itself. The decision to store the journal in an inode contributes to the goal of backwards/forwards compatibility with ext2, and avoids the use of incompatible extensions to the ext2 metadata \citep{Robbins2001a}. It is common for the journal to be stored within the filesystem itself, however it is also possible to store the journal on a separate device or partition \citep{Prabhakaran2005a, Tweedie2000}.

Once the process of journaling is complete, data and metadata are eventually placed into an ext2 structured fixed location on the disk \citep{Prabhakaran2005a}.

\subsection{Journaling Block Device (JBD)}

In order to implement journaling in ext3, an API external to the filesystem was developed, called the \emph{Journaling Block Device} (JBD). Its main purpose is to implement ``a journal on any kind of block device'' \citep[p. 8]{Robbins2001a}. \citet{Tweedie2000} asserts that the JBD is completely encapsulated from ext3. It does not know anything about how the filesystem works, and conversely, the filesystem does not know anything about journaling. The extent of ext3's knowledge of journaling is through \emph{transactions}.

The JBD API allows the filesystem to communicate the modifications to be performed to the JBD as transactions, which in turn is recorded to the journal by the JBD. Ext3 also requests permissions from the JBD before modifying certain data on the disk, providing the JBD the opportunity to manage the journal on behalf of the ext3 filesystem driver \citep{Robbins2001a}.

\subsection{Transactions}

Transactions are the main concept in journaled filesystems, which ``corresponds to a single update of the filesystem'' \citep[p. 4]{Tweedie1998}. This concept has been implemented into ext2 to create the ext3 filesystem and provides a format for the filesystem to communicate with the JBD API \citep{Tweedie2000}.

Similar to database transactions, a journaled filesystem transaction deals with a sequence of changes to the filesystem as a ``single, atomic operation'' \citep[p. 4]{Best2002}. In other words, the entire sequence of a transaction must be completed, or none at all. As \citet[p. 4]{Tweedie2000} explains, ``exactly one transaction results from any single filesystem request made by an application, and contains all of the changed metadata resulting from that request''.

\subsection{Journal approach}

\citet[p. 3]{Robbins2001a} describes two methods of journaling, \emph{logical journaling} and \emph{physical journaling}. Logical journaling refers to the method where the journal stores ``spans of bytes that need to be modified on the host filesystem'' and is found to be used in other journaled filesystems such as XFS. In other words, this approach only records individual bytes that require modification, and is efficient at storing many, smaller modifications to a filesystem \citep{Robbins2001a}.

On the other hand, physical journaling is where entire blocks of modified filesystem are recorded in the journal, and is the journaling approach used by ext3. This means that an unmodified piece of data may also be recorded in the journal, if it resides in a modified block. Although the journal will require more space, it requires less CPU overhead compared to logical journaling, since there is less complexity transferring a literal block from memory to disk \citep{Robbins2001a}.

\subsection{Ext3 journal modes}

Ext3 offers three modes of journal operation, which can be selected during the mounting of the filesystem. These modes are \emph{writeback mode}, \emph{ordered mode}, and \emph{data journaling mode} \citep{Prabhakaran2005a, Jones2008}. Each mode offers varying levels of data consistency guarantees and are interchangeable.

\subsubsection{Writeback mode}

In writeback mode, only metadata is journaled and data blocks are written directly to disk. While this mode preserves the filesystem structure and guarantees metadata consistency, it provides the ``weakest consistency semantics of the three modes'' \citep[p. 108]{Prabhakaran2005a}. In other words, it is still possible for data to be corrupted, because the order between journal and fixed location data writes are not enforced. For example, if a system crash occurs after metadata has been journaled, but before the data block is written, it is likely the data may contain garbage or previously written data \citep{Jones2008, Prabhakaran2005a}. However, this mode provides ``the best ext3 performance under most conditions'' \citep[p. 2]{Robbins2001b}.

\subsubsection{Ordered mode}

Similar to writeback mode, ordered mode only journals metadata. However, order between journal and fixed location data writes are enforced. This is the default mode, if a user does not select one during the mounting of the ext3 filesystem. Metadata and data block writes are grouped logically as transactions, as mentioned in Section 4.2. When the time comes to \emph{flush} the transaction (write metadata) to disk, data blocks must be written first before metadata is journaled \citep{Robbins2001b}. This ordering of writes effectively guarantees both metadata and data recovery consistency \citep{Jones2008, Prabhakaran2005a}.

\subsubsection{Data journaling mode}

Data journaling mode offers the guarantee of data and metadata consistency, due to the journaling of both metadata and data. However, there are performance trade-offs with this mode since data is being written twice: once to the journal, and again to the fixed ex2 location. Data journaling mode is generally considered the slowest of all ext3 journaling modes, however \citet[p. 3]{Robbins2001b} references an experiment which shows otherwise.