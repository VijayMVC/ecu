\section{Ext3 journal mechanics}

\citet[p. 5]{Tweedie1998} explains that a journal has two concurrent responsibilities: ``record the new content of filesystem metadata blocks while [it is] in the process of committing transactions''. This sentiment is also echoed by \citet[p. 55]{Galli2001}. With backwards/fowards compatibility in mind, journaling functionality has been added to ext2 to create the ext3 filesystem by implementing two main concepts, the Journaling Block Device, and transactions, providing functionality to satisfy the two responsibilities of the journal.

The ext3 journal itself is stored in an inode using a circular buffer data structure \citep{Robbins2001a, Tweedie2000, Jones2008, Prabhakaran2005a}. An inode is an area on the disk where all the information about a file is stored, however, the actual file itself is not stored in an inode. \citet[p. 2]{Best2002} uses the analogy of a ``bookkeeping file for a file'', and identifies that an inode is in fact a file itself. The decision to store the journal in an inode contributes to the goal of backwards/forwards compatibility with ext2, and avoids the use of incompatible extensions to the ext2 metadata \citep{Robbins2001a}. It is common for the journal to be stored within the filesystem itself, however it is also possible to store the journal on a separate device or partition, and even have multiple filesystems sharing the same journal \citep{Prabhakaran2005a, Tweedie2000, Galli2001}.

Additionally, the circular buffer could be described as a temporary holding area for the journal. Space is cleared and reclaimed in the buffer for the journal, once data and metadata are placed into their fixed locations on disk \citep{Prabhakaran2005a}. The sections to follow describe the various implemented mechanics which make it possible for ext3 to function as a journaled filesystem.

\subsection{Journaling Block Device (JBD)}

In order to enable journaling in ext3, an API external to the filesystem was developed, called the \emph{Journaling Block Device} (JBD). Its main purpose is to implement ``a journal on any kind of block device'', acting as an external layer to provide ext3 with journaling capabilities \citep[p. 8]{Robbins2001a, Mauerer2008}. \citet{Tweedie2000} asserts that the JBD is completely encapsulated from ext3. It does not know anything about how the filesystem works, and conversely, the filesystem does not know anything about journaling. The extent of ext3's knowledge of journaling is through \emph{transactions}.

The JBD API allows the filesystem to communicate the modifications to be performed to the JBD as transactions, which in turn is recorded to the journal by the JBD. In order for the JBD to have an opportunity to manage the journal on the ext3 filesystem driver's behalf, ext3 requests permissions from the JBD before modifying certain data on the disk \citep{Robbins2001a}.

\subsection{Journal approach}

\citet[p. 3]{Robbins2001a} describes two methods of journaling, \emph{logical journaling} and \emph{physical journaling}. Logical journaling refers to the method where the journal stores ``spans of bytes that need to be modified on the host filesystem'' and is found to be used in other journaled filesystems such as XFS. In other words, this approach only records individual bytes that require modification, and is efficient at storing many, smaller modifications to a filesystem \citep{Robbins2001a}.

On the other hand, physical journaling is where entire blocks of modified filesystem are recorded in the journal, and is the journaling approach used by ext3. This means that when a small change is made, the entire journal block must be written \citep{Bovet2006, Robbins2001a}. Although the journal will require more space, it requires less CPU overhead compared to logical journaling, since there is less complexity transferring a literal block from memory to disk \citep{Robbins2001a}, and ``journal operations can be batched into large clusters'' \citep[p. 55]{Galli2001}.

\subsection{Journal hierarchy}

The contents of the journal form a hierarchy of information, starting with log records which record the lowest level of filesystem operation. Log records are then grouped by handles, which represent operations at high level system calls. Finally, transactions group handles together and represent a single, atomic update to the filesystem.

\subsubsection{Log records}

A log record is an individual low level operation to update a block and is the smallest unit of data that can be logged in the journal \citep{Mauerer2008}. As discussed previously, ext3's journal approach is physical journaling, where entire blocks of modified filesystem are recorded in the log. Therefore, an ext3 log record consists of ``the whole buffer modified by the low level operation'' \citep[p. 770]{Bovet2006}. Each log record contains a \mintinline{console}{journal_block_tag_t}, storing the associated logical block number within the filesystem as well as status flags \citep{Bovet2006}.

\subsubsection{Handles}

A handle is a group of low level log records that represent a high level atomic operation at the system level. \citet[p. 639]{Mauerer2008} explains that if a \mintinline{console}{write} system call is invoked, any log records associated with that particular call would be grouped together to form a handle. Therefore, each system call must be dealt with in an atomic manner to prevent data corruption, and is assisted by recording handles in the journal. In other words, similar to transactions, a handle must perform the entire sequence of log record changes to disk, or none at all. Each handle is ``represented by a descriptor of type \mintinline{console}{handle_t} \citep[p. 770]{Bovet2006}.

\subsubsection{Transactions}

Transactions are the main concept in journaled filesystems, which ``corresponds to a single update of the filesystem'' \citep[p. 4]{Tweedie1998}. Transactions are formed by groups of handles \citep{Mauerer2008, Bovet2006}, and log records that related to a particular handle must reside in a single transaction. This concept has been implemented into ext2 to create the ext3 filesystem and provides a format for the filesystem to communicate with the JBD API \citep{Tweedie2000}, and as a means of efficiency for the filesystem \citep{Bovet2006}.

Similar to database transactions, a journaled filesystem groups together ``collections of atomic sequence of events'' and is executed as a single operation, rather than taking each individual filesystem into account \citep[p. 2]{Katiyar2011}. In other words, the entire sequence of a transaction must be completed, or none at all. As \citet[p. 4]{Tweedie2000} explains, ``exactly one transaction results from any single filesystem request made by an application, and contains all of the changed metadata resulting from that request''.

Each transaction contains a descriptor \mintinline{console}{transaction_t}, which in turn contains field \mintinline{console}{t_state}, describing the status of the transaction \citep{Bovet2006}. The state of a transaction determines whether or not it is considered during recovery after an improper shutdown. \citet[p. 771]{Bovet2006} describes each of the possible states of a transaction:

\paragraph{Complete}

A transaction is marked complete when all associated log records have been physically written to the journal and are considered to assist in recovery after an improper shutdown. The \mintinline{console}{t_state} field contains value \mintinline{console}{T_FINISHED} when a transaction is completed.

\paragraph{Incomplete}

A transaction is considered incomplete when not all associated log records have been physically written to the journal, or subsequent log records are still being added to the transaction. Incomplete transactions are discarded during recovery after an improper shutdown. Incomplete transactions have four possible values stored in the \mintinline{console}{t_state} field:

\begin{table}[H]
\centering
\caption{Incomplete \mintinline{console}{t_state} values \citep[p. 771-772]{Bovet2006}.}
\begin{tabular}{l|l}
Value							& Desciption\\ \hline
\mintinline{console}{T_RUNNING}	& Transaction is still accepting new handles \\
\mintinline{console}{T_LOCKED}	& Transaction no longer accepting new handles, but some are unfinished \\
\mintinline{console}{T_FLUSH}	& Transaction has recorded all handles, but is still in the process of \\ & recording remaining log records \\
\mintinline{console}{T_COMMIT}	& Transaction has finished writing all associated log records of \\ & handles to disk, but transaction is yet to be marked completed \\ & on the journal
\end{tabular}
\end{table}

The journal may contain multiple transactions, but only one transaction may be active at any given time, with state \mintinline{console}{T_RUNNING}. The other transactions contained in the journal may be incomplete, waiting for buffers to write the associated log records to the journal \citep{Bovet2006}.

A transaction is marked complete when all associated log records have been written to the journal, but updates to the fixed location on disk may still not have occurred. A complete transaction is only removed from the journal after verification that a complete transaction has been written to the fixed location on disk \citep{Bovet2006}. This process is handled through checkpointing, discussed in a later section.

\subsection{Journal structure}

As each ext3 transaction is recorded into the journal, they are organized and described by the following metadata blocks: \emph{journal superblock}, \emph{descriptor block}, and \emph{journal commit block} \citep{Prabhakaran2005a}. Summary information such as block size and head/tail pointers are recorded by the journal superblock. As discussed previously, ext3 logs an entire block to journal if it has even the smallest update required, and summary information about these blocks are stored in the journal superblock \citep{Prabhakaran2005a, Galli2001}

The start of a transaction is marked by the descriptor block, containing information about the subsequent journaled blocks as well as their fixed locations on disk \citep{Prabhakaran2005a}. The role of descriptor blocks is to describe other metadata journal blocks, supporting journaled recovery functions \citep{Galli2001, Tweedie1998}. What is stored after the descriptor block in a journal depends on the selected ext3 mode of operation, as described in more detail below. In data journaling mode, data and metadata blocks are stored after the descriptor block. In writeback and ordered modes, metadata blocks are stored after the descriptor block \citep{Prabhakaran2005a}.

The journal commit block signals the end of a transaction, and is stored after the metadata and data blocks marked for update in that particular transaction. ``Once the commit block is written, the journaled data can be recovered without loss'' \citep[p. 109]{Prabhakaran2005a}.

\subsection{Ext3 journal modes}

Ext3 offers three modes of journal operation, which can be selected during the mounting of the filesystem. These modes are \emph{writeback mode}, \emph{ordered mode}, and \emph{data journaling mode} \citep{Prabhakaran2005a, Jones2008, Mauerer2008}. Each mode offers varying levels of data consistency guarantees and are interchangeable.

\subsubsection{Writeback mode}

In writeback mode, only metadata is journaled and data blocks are written directly to disk. While this mode preserves the filesystem structure and guarantees metadata consistency, it provides the ``weakest consistency semantics of the three modes'' \citep[p. 108]{Prabhakaran2005a}. In other words, it is still possible for data to be corrupted, because the order between journal and fixed location data writes are not enforced. For example, if a system crash occurs after metadata has been journaled, but before the data block is written, it is likely the data may contain garbage or previously written data \citep{Jones2008, Prabhakaran2005a}. However, this mode provides ``the best ext3 performance under most conditions'' \citep[p. 2]{Robbins2001b}.

\subsubsection{Ordered mode}

Similar to writeback mode, ordered mode only journals metadata. However, order between journal and fixed location data writes are enforced. This is the default mode, if a user does not select one during the mounting of the ext3 filesystem. Metadata and data block writes are grouped logically as transactions, as mentioned previously. When the time comes to \emph{commit} the transaction (write metadata) to disk, data blocks must be written first before metadata is journaled \citep{Robbins2001b}. This ordering of writes effectively guarantees both metadata and data recovery consistency \citep{Jones2008, Prabhakaran2005a}.

\subsubsection{Data journaling mode}

Data journaling mode offers the guarantee of data and metadata consistency, due to the journaling of both metadata and data. However, there are performance trade-offs with this mode since data is being written twice: once to the journal, and again to the fixed ex2 location. Data journaling mode is generally considered the slowest of all ext3 journaling modes, however \citet[p. 3]{Robbins2001b} references an experiment which shows it can perform well where ``interactive performance IO needs to be maximized''.

\subsection{Checkpointing}

After a transaction has been committed, the data at the corresponding fixed location on disk is yet to be updated. Therefore, the corresponding data in the journal must remain in the journal until the data at the fixed locations have been synced to the corresponding transactions in the journal \citep{Tweedie1998}. This syncing process is called checkpointing.

A task handled by the JBD, checkpointing involves writing all the contents of the journal to their respective fixed locations on disk as a compound transaction. Once data is copied to the fixed location on disk, the compound transaction is completed and the space taken up by the checkpointed compound transaction on the journal can be reclaimed by subsequent transactions \citep{Tweedie2000, Tweedie1998, Prabhakaran2005a, Katiyar2011, Devlisden2011}. Various events can trigger the checkpointing process, such as low space in either the filesystem buffer or journal itself, or when a time-out expires \citep{Prabhakaran2005a}.



