\section{Roots in ext2}

In addition to creating a journaled filesystem for Linux, \citet{Tweedie2000} describes one of the main goals in the design of the ext3 filesystem is the compatibility with the existing ext2 filesystem. This goal was justified, as much of the Linux user base was using ext2 at the time, and greater compatibility between the two filesystems would ensure seamless transition for users. This goal was successfully achieved, and an ext3 filesystem that is cleanly unmounted can easily be remounted as an ext2 filesystem. Conversely, an ext2 filesystem can easily be upgraded to ext3 \emph{in-place}, while the volume is mounted \citep{Tweedie2000, Robbins2001a, Bovet2006}.

This backwards and forwards compatibility was achieved by using the same source code base for ext3 as ext2. Therefore, ext2 and ext3 share many common properties. For example, both filesystems use the same on-disk and metadata format. Ext3 also inherits ext2's \mintinline{console}{fsck} \citep{Robbins2001a, Tweedie2000}. Although the desired effect of ext3 is to avoid the use of \mintinline{console}{fsck}, there are still times when it may be required, in the event that metadata is corrupted. Availability of \mintinline{console}{fsck}, with years of development further enhances the reliability of ext3, when compared to other journaled filesystems which do not have a similar capability \citep{Robbins2001a}.