\section{Learnings}

Due to the greater amount of input sensors provided, this controller is vastly different to the one submitted with the first assignment. It is now possible to define rules with higher complexity, and therefore 3D rule matrices are a necessity. The larger player count also requires a completely different strategy. In the first assignment, the main strategy was to be aggressive from the start. In comparison, this controller needs to be much more defensive and aim to survive until the end of the battle.

With that in mind, I began with blast sensor inputs. Rather than attempt to implement multiple sensors at once, I was able to focus on the controller's performance on dodging energy blasts and developed a well-performing ruleset just for turning. Experience from the first assignment assisted in defining controlled turn rules, resulting in less erratic movement. I reused the clock analogy for target/blast/powerup directions, and also considered the heading of moving objects to decide which direction to turn the player. 

However, I came across an issue where the controller would switch between -180 and +180 when an energy blast was behind the player. Rules dictated that the player would turn either left or right, depending on if the energy blast position was a positive or negative number. Because the sensor value was switching between positive and negative values erratically, the controller also flicked between left and right turns, resulting in the player not turning at all, and continued to move straight and was always hit by energy blasts from behind. To remedy this, I decided to stick to a single direction of turn, which in this case was a right hand 90$^{\circ}$ turn. This resolved the issue and the player now dodges energy blasts very well from the 6 o'clock position, albiet only in right hand turns.

I encountered an issue where I wished to prioritize turn rules that tell the player to move directly to a powerup over turn rules that dodge energy blasts. I emailed the tutor, Philip Hingston, who advised me that this was not possible. The alternative was to set up a combination of fuzzy and if-then rules. I implemented this concept and realised this opened up other possibilities for rules, as seen with the defensive and offensive versions of my output variables. I found similarities in this concept with the Tartarus problem of Workshop 7, which assisted in designing the if-then rules. I can see the potential that if-then statements have in developing richer, more complex rules.

The controller has potential to be improved. For example, firepower rules could utilize target heading, and only fire if the target is facing the player. I observe that the player can sometimes waste shots if the target is running away and does a good job at dodging energy blasts. The turn rules also have room for improvement. There are some situations where the player makes a wrong turn, or does not react quick enough to an incoming energy blast. I am pleased with the powerup strategy I have implemented, as it ignores energy blasts and suppresses enemies nearby while heading straight for the powerup. However, there are cases where the nearest enemy to the player is not the one closest to the powerup, and energy is wasted firing at the maximum rate. This could be improved by determining which enemy is closest to the powerup and firing at it, or by having the ability to fire at the powerup directly.