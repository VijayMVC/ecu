\newpage

\section{Learnings}

Although I believe I have a firm grasp on the basic concepts of fuzzy logic, I had a considerable amount of trouble understanding the logic behind \emph{turning} the saucer. I ran several tests, printing the opponent direction values during runtime to understand the relationship between where the player saucer is, and where the enemy is, and the values that were returned each frame. It was only until I imagined that the positive, left-hard turn values as ``counter-clockwise'', and negative, right-hand turn values as ``clockwise'' did I manage to tune the turn rules to a working state.

My initial concept was to aggressively attack the opponent as soon as the game started, which worked well. However, the existing opponent controllers did not test my defensive strategy at all, and were immediately overwhelmed by incoming fire. My saucer was always within \emph{winning} or \emph{even} fuzzy set limits, and never triggered \emph{losing} rules. Eventually, I realised that the original \mintinline{console}{FuzzyController.java} class could be modified to become a more challenging opponent by changing the values, as listed below:

\begin{listing}[H]
\caption{FuzzyController.java modifications}

\begin{javacode}
public FuzzyController() throws Fuzzy Exception {
  // ...
  final double maxPower = Saucer.MAX_POWER;
  final double midPower = maxPower; // originally divided by 5.0
  final double lowPower = maxPower; // originally divided by 20.0
  // ...
}
\end{javacode}
\end{listing}

After the changes were made, the \emph{fuzzy} opponent fired its cannons at maximum power, which assisted in testing whether or not my offensive strategy to constantly stay within close range to the enemy worked. Initial tests proved otherwise, and my saucer was destroyed immediately. I needed to develop a substantial defensive strategy.

The \emph{heading angle} fuzzy sets were modified from my original, arbitrarily chosen sets, to sets that are based on clock positions in relation to the player's position, similar to what a fighter pilot might say during combat, ie. ``Bandit at my six o'clock'', or ``Bogey at my nine''. The \emph{turn} output spikes were also chosen based on the clock analogy, resulting in a more controlled behaviour, and enabled me to define a defensive strategy through \emph{turn} rules.

However, the rules require refinement. Confusion can occur when the player attempts to travel beyond the battle space limits to maintain its heading and causes erratic turns. This may be resolved if sensors were in place to detect the battle space boundaries, and added as linguistic variables and rules. Confusion can also occur when too many \emph{heading angle} antecedents fire rules which result in twitchy turning movements. This would be a major problem if turns cost energy. \emph{Firepower} also has the potential for improvement as it may occasionally fire weak shots even though the enemy is further than the shot's range, resulting in wasted energy.

My offensive strategy is also far from perfect and has much room for improvement. Sitting right behind the enemy and giving chase during offence may suit well to space/aircraft with fixed, forward facing armament, but is a dangerous place to be when the enemy can rotate his weapon to face the rear.