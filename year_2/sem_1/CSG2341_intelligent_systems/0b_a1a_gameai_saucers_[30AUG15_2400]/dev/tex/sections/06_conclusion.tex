\newpage

\section{Conclusion}

This report examined the application of fuzzy logic using Sugeno style inference in a video game. The video game involved developing a fuzzy logic controller, which controls an armed flying saucer with the sole purpose of destroying the enemy saucer.

The linguistic input variables and their fuzzy sets were explored, as well as the output variables and the associated rules. These rules governed the behaviour of the saucer and attempted to implement the two overall tactics developed for this assignment:

\begin{itemize}
	\item Commit to the battle and fly offensively when the score is even or if winning
	\item Disengage from the battle and fly defensively when losing
\end{itemize}

An example rule was examined, and the aggregation of a crisp output was explained, based on Sugeno style inference. Finally, learning experiences throughout the work of this assignment were discussed.

\subsection{Results}

The following tables on the final page display the results of tournaments between the \emph{checkSix} controller versus the \emph{simple} controller, \emph{fuzzy} controller, as well as the modified fuzzy controller, as discussed in Section 5. The modified fuzzy controller is referred to as \emph{fuzzyMaxPower} in the table. The average score over 10 battles per tournament is shown, with five consecutive tournaments executed per opponent.

The final table shows the results when \emph{checkSix} goes head to head with itself. No changes were made to the \emph{evilCheckSix} controller, other than its name and colour. It was expected that the scores would have been more equal. Surprisingly, that is not the case, although it is assumed that further tournament executions may balance out the score averages.

\newpage

\begin{table}[H]
\centering
\caption{\emph{checkSix} vs. \emph{simple}}
\label{checkSix vs. simple}
\begin{tabular}{r|r|r}
Tournament	& checkSix	& simple	\\ \hline
1			& 3568.44	& 0.0		\\
2			& 3813.32	& 0.0		\\
3			& 3518.27	& 0.0		\\
4			& 3433.35	& 0.0		\\
5			& 3181.94	& 0.0
\end{tabular}
\end{table}

\begin{table}[H]
\centering
\caption{\emph{checkSix} vs. \emph{fuzzy}}
\label{checkSix vs. fuzzy}
\begin{tabular}{r|r|r}
Tournament	& checkSix	& fuzzy	\\ \hline
1			& 2586.32	& 0.0	\\
2			& 2709.84	& 0.0	\\
3			& 2652.23	& 0.0	\\
4			& 2245.64	& 0.0	\\
5			& 2769.11	& 0.0
\end{tabular}
\end{table}

\begin{table}[H]
\centering
\caption{\emph{checkSix} vs. \emph{fuzzyMaxPower}}
\label{checkSix vs. fuzzyMaxPower}
\begin{tabular}{r|r|r}
Tournament	& checkSix	& fuzzyMaxPower	\\ \hline
1			& 83.05		& 80.34			\\
2			& 67.07		& 50.50			\\
3			& 75.71		& 94.48			\\
4			& 29.69		& 61.04			\\
5			& 115.98	& 30.90
\end{tabular}
\end{table}

\begin{table}[H]
\centering
\caption{Bonus round \emph{checkSix} vs. \emph{evilCheckSix}}
\label{Bonus round checkSix vs. evilCheckSix}
\begin{tabular}{r|r|r}
Tournament	& checkSix	& evilCheckSix	\\ \hline
1			& 109.72	& 69.16			\\
2			& 140.43	& 54.01			\\
3			& 194.95	& 39.28			\\
4			& 898.02	& 2.83			\\
5			& 406.34	& 7.17
\end{tabular}
\end{table}