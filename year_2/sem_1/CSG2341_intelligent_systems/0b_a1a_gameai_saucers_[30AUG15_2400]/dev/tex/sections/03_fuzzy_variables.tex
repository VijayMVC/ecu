\newpage

\section{Fuzzy variables}

Three linguistic variables have been selected for input, based on the sensor readings of the saucer:

\begin{itemize}
	\item distance
	\item energyDifference
	\item headingAngle
\end{itemize}

\subsection{distance}

Distance is the distance from the saucer to the opponent, and is measured in meters. The universe of disclosure for distance is between 0 meters and the diagonal length of the battle space. The formula has been supplied in the existing code as:
\\
\\
$\sqrt{\mbox{width} \cdot \mbox{width} + \mbox{height} \cdot \mbox{height}}$
\\
\\
This linguistic variable is used to determine how much energy will be committed to firing the cannon. As mentioned previously, the cannon will only be fired at close or near distances. Therefore, three fuzzy sets are associated with distance.

\begin{itemize}
	\item close
	\item near
	\item far
\end{itemize}

\subsection{energyDifference}

Energy difference is the difference between the saucer's energy and the opponent's energy. The universe of disclosure for energyDifference is between -10,000j to +10,000j, where 10,000j is the amount of energy that the saucers begin with. This linguistic variable determines who is winning, who is losing, or if the score is even. This variable is used as input to decide how much energy is committed in firing the weapon, as well as whether or not to turn into or away from the enemy. The following fuzzy sets are created for energyDifference:

\begin{itemize}
	\item losing
	\item even
	\item winning
\end{itemize}

\subsection{headingAngle}

Heading angle is the direction of the opponent in relation to the saucer. These values