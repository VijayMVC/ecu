\section{Introduction}

This report examines array, singly linked list and binary search tree data structures and some algorithms which interact with them. Implementation of these data structures and algorithms in Java will be outlined in the report, along with analysis in big O notation of algorithms used. 

Throughout this report, it is assumed that the selection of algorithms is based on time efficiency rather than space efficiency. In other words, an algorithm with higher time efficiency but lower space efficiency will be selected over an algorithm with a lower time efficiency and higher space efficiency.

The array data structure is demonstrated through the implementation of a simple lotto game. The lotto game allows up to 1000 players, with each player picking six unique integers, between 1 and 45. Each player and their respective lotto tickets are represented inside a two-dimensional array, while the winning numbers are represented inside a one-dimensional array. Merge sort is used to sort player tickets and winning numbers arrays, while binary search and an adaptation of an array merge algorithm is used to determine if a player's ticket contains the winning numbers.

The singly linked list data structure is demonstrated through an existing implementation of a list of students and their marks for a particular unit. The existing class has been modified to include methods to delete a particular student from the list, and to print the ascending ordered list in reverse, descending order. An issue with permanently deleting the first node was experienced with the existing class, and re-written/re-designed classes have been provided in the source files, attempting to resolve the issue.

The binary search tree data structure is demonstrated through an existing binary search tree of random integers. An existing class has been modified to include methods which print all \emph{leaf} nodes of a tree, print all \emph{non-leaf} nodes of a tree, and to calculate the height/depth of a tree.