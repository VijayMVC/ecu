\newpage
\section{Arrays}

The array data structure is demonstrated through the implementation of a simple lotto game. The lotto game allows up to 1000 players, with each player picking six unique integers, between 1 and 45, which makes up their lotto ticket. Each player and their respective lotto tickets are represented inside a two-dimensional array, while the winning numbers are represented inside a one-dimensional array.
\\
\\
The following classes have been created to represent the lotto game:

\begin{itemize}
\item \mintinline{java}{Main}: The executable class, contains \mintinline{java}{main()} method
\item \mintinline{java}{PlayerTickets}: Generates a two-dimensional array which represents each player, and their picks for the lotto ticket
\item \mintinline{java}{WinningNumbers}: Generates a one-dimensional array which represents the winning numbers for the lotto game
\item \mintinline{java}{WinningPlayers}: Contains logic to determine who the winning players are, which of their numbers match the winning numbers, and their winner class category
\item \mintinline{java}{Sorter}: Helper class to sort \mintinline{java}{PlayerTickets} and \mintinline{java}{WinningNumbers} arrays
\item \mintinline{java}{Randomizer}: Helper class to generate random numbers for each player pick and winning number
\end{itemize}

\subsection{Sorting}

In order to use more efficient search algorithms such as binary search, the arrays must be sorted first. The merge sort algorithm has been selected due to its time efficiency of $O(n \ log \ n)$. This algorithm is implemented as a static method of the \mintinline{Java}{Sorter} class, as shown in Java code 2.1 and 2.2.

\subsubsection{Merge sort algorithm}

To sort $a$[left...right] into ascending order:

\begin{enumerate}
\item If left $\leq$ right:
	\begin{enumerate}
	\item Let $m$ be an integer about midway between left and right
	\item Sort $a$[left...$m$] into ascending order
	\item Sort $a$[$m+1$...right] into ascending order
	\item Merge $a$[left...$m$] and $a$[$m+1$...right] into auxiliary array $b$
	\item Copy all components of $b$ into a[left...right]
	\end{enumerate}
\item Terminate
\end{enumerate}

\noindent
\citep[p. 54]{Watt2001}


\subsubsection{Merge sort Java method}

\begin{listing}[H]
\caption{Merge sort method}
\begin{javacode}
private static void mergeSort(int low, int high) {

    // 1.0 If left (low) < right (high)
    if(low < high) {

        // 1.1 Let m (mid) be an integer about midway between left and right
        int mid = low + (high - low) / 2;

        // 1.2 Sort a[left...m] into ascending order
        mergeSort(low, mid);

        // 1.3 Sort a[m+1...right] into ascending order
        mergeSort(mid + 1, high);

        // 1.4 Merge a[left...m] and a[m+1...right] into auxiliary array b
        // call merge() which is O(n)
        merge(low, mid, high);
    }
}
\end{javacode}
\end{listing}

\noindent
At line 17, the \mintinline{java}{mergeSort()} method calls supporting method \mintinline{java}{merge()}, as shown in Java code 2.2, in order to perform step 1.4 of the merge sort algorithm.

\begin{listing}[H]
\caption{Merge method}
\begin{javacode}
private static void merge(int low, int mid, int high) {

    // iterate from low through to high
    for(int i = low; i <= high; i++) {

        // copy each element from the array to sort, to each element into temp array
        mergeTempArray[i] = mergeArrayToSort[i];
    }

    // 1.0 Set i = low, set j = mid + 1, set k = low
    int i = low;
    int j = mid + 1;
    int k = low;

    // 2.0 While i <= mid AND j <= high, repeat:
    while(i <= mid && j <= high) {

        // 2.1 If mergeTempArray[i] <= mergeTempArray[j],
        if(mergeTempArray[i] <= mergeTempArray[j]) {

            // 2.1.1 Copy mergeTempArray[i] into mergeArrayToSort[k], then increment i and k
            mergeArrayToSort[k] = mergeTempArray[i];
            i++;

        // 2.2 If mergeTempArray[i] > mergeTempArray[j],
        } else {

            // 2.2.1 Copy mergeTempArray[j] into mergeArrayToSort[k], then increment j and k
            mergeArrayToSort[k] = mergeTempArray[j];
            j++;
        }
        k++;
    }

    // 3.0 While i <= mid,
    while(i <= mid) {

        // 3.1 Copy mergeTempArray[i] into mergeArrayToSort[k], then increment i and k
        mergeArrayToSort[k] = mergeTempArray[i];
        k++;
        i++;
    }
}
\end{javacode}
\end{listing}

\subsubsection{Merge sort analysis}

As \citet[p. 54 - 55]{Watt2001} explain, analysis of the merge sort algorithm's time complexity involves counting the number of comparisons made during the operation. Let $n =$ right $-$ left $+ 1$ be the length of the array, and let $C(n)$ be the total number of comparisons required to sort $n$ values.

Step 1.1 involves dividing the array into two subarrays, $n/2$. The left subarray takes around $C(n/2)$ comparisons to sort, and similarly, the right subarray takes around $C(n/2)$ comparisons to sort.

Step 1.4 involves the merging of each subarray into a sorted array and takes about $n - 1$ comparisons to complete. Therefore:

\begin{equation}
C(n) \approx \left\{\begin{matrix}
2C(n/2) + n - 1 & \mbox{if} \ n > 1 \\ 
0 & \mbox{if} \ n \leq 1 
\end{matrix}\right.
\end{equation}

\newpage
\noindent
Simplifying equation 2.1:

\begin{equation}
C(n) \approx n \times \log_2n
\end{equation}

\noindent
Therefore the time complexity is $O(n \log n)$.
\\
\\
Space complexity is $O(n)$, since step 1.4 requires an auxiliary array of length $n$ to temporarily store the sorted array.

\subsubsection{Merge sort console output}

The following console outputs demonstrate that the merge sort algorithm is functioning correctly, and sorts player lotto picks and winning numbers in ascending order as desired. Note that the examples shown truncate actual results down to ten results from a thousand.
\\
\begin{consolecode}
***********************
*** UNSORTED ARRAYS ***
***********************

Player 0001 picks: [16][14][37][31][07][42]
Player 0002 picks: [20][12][26][23][44][16]
Player 0003 picks: [10][32][20][35][02][24]
Player 0004 picks: [06][23][19][35][42][25]
Player 0005 picks: [07][17][28][41][29][38]
Player 0006 picks: [16][05][36][31][04][23]
Player 0007 picks: [20][01][34][37][07][18]
Player 0008 picks: [30][29][10][27][34][05]
Player 0009 picks: [03][27][13][38][28][32]
Player 0010 picks: [08][24][45][14][02][07]

Winning Numbers:   [21][22][10][03][04][13]
\end{consolecode}

\begin{consolecode}
***********************
**** SORTED ARRAYS ****
***********************

Player 0001 picks: [07][14][16][31][37][42]
Player 0002 picks: [12][16][20][23][26][44]
Player 0003 picks: [02][10][20][24][32][35]
Player 0004 picks: [06][19][23][25][35][42]
Player 0005 picks: [07][17][28][29][38][41]
Player 0006 picks: [04][05][16][23][31][36]
Player 0007 picks: [01][07][18][20][34][37]
Player 0008 picks: [05][10][27][29][30][34]
Player 0009 picks: [03][13][27][28][32][38]
Player 0010 picks: [02][07][08][14][24][45]

Winning Numbers:   [03][04][10][13][21][22]
\end{consolecode}

\newpage
\subsection{Searching}

In order to determine the winners of the lotto game, the program must be able to search each number from the winning numbers array within each array of player tickets. Total number of winners within each winner class category must be calculated and displayed, as well as the result for an individual player, which simulates a player requesting their ticket to be checked for winning numbers.

For both functions to be implemented, two search algorithms have been selected, binary search and an adaptation of the merge array algorithm to sequentially compare components of two sorted arrays. Both of these algorithms may be implemented since the arrays have been sorted in ascending order. These algorithms are implemented as instance methods within the \mintinline{java}{WinningPlayers} class, as shown in Java code 2.3 and 2.4.

\subsubsection{Binary search algorithm}

To find which (if any) component of the sorted (sub)array $a$[left...right] equals target:

\begin{enumerate}
\item Set $l$ = left, $r$ = right
\item While $l \ \leq \ r$, repeat:
	\begin{enumerate}
	\item Let $m$ be an integer about halfway between $l$ and $r$
	\item If target equals $a[m]$, terminate with answer $m$
	\item If target is less than $a[m]$, set $r = m - 1$
	\item If target is greater than $a[m]$, set $l = m + 1$
	\end{enumerate}
\item Terminate with answer none
\end{enumerate}

\noindent
\citep[p. 43]{Watt2001}

\subsubsection{Binary search Java method}

\begin{listing}[H]
\caption{Binary search method}
\begin{javacode}
private int binarySearch(int[] array, int target) {

    // 1.0 Set l = left, and r = right (substituted with low and high respectively)
    int low = 0;
    int high = array.length - 1;

    // 2.0 While l <= r, repeat:
    while(low <= high) {

        // 2.1 Let m (mid) be an integer about midway between l and r
        int mid = low + (high - low) / 2;

        // 2.2 If target equals a[m], terminate with answer m
        if(target == array[mid]) {
            return mid;

        // 2.3 If target is less than a[m], set r = m - 1
        } else if(target < array[mid]) {
            high = mid - 1;

        // 2.4 If target is greater than a[m], set l = m + 1
        } else {
            low = mid + 1;
        }
    }

    // 3.0 Terminate with answer none
    return -1;
}
\end{javacode}
\end{listing}

\subsubsection{Binary search analysis}

Analysis of the binary search algorithm's time complexity involves counting the number of comparisons made during the operation. Let $n =$ right $-$ left $+ 1$ be the length of the array, and assume that steps 2.2 to 2.4 are implemented as a single comparison. At most, these steps are iterated as many times as $n$ can be halved until it reaches 0. Therefore the number of comparisons is $\mbox{floor}(\log_2n) + 1$ \citep{Watt2001}. The time complexity for binary search is $O(\log n)$.

\newpage
\subsubsection{Binary search console output}

The console output below displays the total number of winners in each winner class. To be considered a winner in this lotto game, a player must match at least 3 picks in their ticket to the winning numbers.

Each winner class is categorised by the number of matches a player may have between the picks in their ticket, and the winning numbers:

\begin{itemize}
\item 1st class: 6 picks match winning numbers
\item 2nd class: 5 picks match winning numbers
\item 3rd class: 4 picks match winning numbers
\item 4th class: 3 picks match winning numbers
\end{itemize}

\begin{consolecode}
***********************
**** BINARY METHOD ****
***********************

1st class winners: 0
2nd class winners: 0
3rd class winners: 4
4th class winners: 26
\end{consolecode}

\noindent
The console output below displays results for individual players.
\\
\begin{consolecode}
** BINARY TICKET CHECKING **

Player 0005 did not win. Thanks for playing lotto. 
Better luck next time!

Player 0500 is a 4th class winner!
Your winning numbers are: [11][33][45]

Player 0564 did not win. Thanks for playing lotto. 
Better luck next time!

Player 0897 did not win. Thanks for playing lotto. 
Better luck next time!
\end{consolecode}

\noindent
The console output below displays the result after a user inputs their player number.
\\
\begin{consolecode}
************************
****** USER INPUT ******
************************

ENTER YOUR PLAYER NUMBER TO CHECK IF YOU HAVE A WINNING TICKET:
500
** BINARY METHOD TICKET CHECK **
Player 0500 is a 4th class winner!
Your winning numbers are: [11][33][45]
\end{consolecode}

\newpage
\subsubsection{``Merge search'' algorithm}

The following algorithm is an adaptation of the arrays merging algorithm. Rather than merge two sorted arrays into one sorted array, the algorithm is used to sequentially compare components from two sorted arrays to find a match between player lotto tickets and winning numbers.
\\
\\
To find which (if any) component from $a1[l1...r1]$ equals any component from $a2[l2...r2]$:

\begin{enumerate}
\item Set $i = l1$, set $j = l2$, let matchTally be the number of matches found, let playerMatchString record the matching components
\item While $i \leq r1$ and $j \leq r2$, repeat:
	\begin{enumerate}
	\item If $a1[i] < a2[j]$:
		\begin{enumerate}
		\item Increment $i$
		\end{enumerate}
	\item If $a1[i] > a2[j]$:
		\begin{enumerate}
		\item Increment $j$
		\end{enumerate}
	\item If $a1[i] ==  a2[j]$:
		\begin{enumerate}
		\item Increment matchTally
		\item Add matching component to playerMatchString
		\end{enumerate}
	\end{enumerate}
\item Terminate with answer matchTally
\end{enumerate}

\noindent
Adaptation of merging algorithm by \citet[p. 46]{Watt2001}.

\subsubsection{``Merge search'' Java method}

\begin{listing}[H]
\caption{``Merge search'' method}
\begin{javacode}
private int mergeSearch(int[] playerTicket, int[] winningNumbers) {

    // 1.0 Set i = l1, set j = l2

    // i tracks playerTicket left
    int i = 0;
    // j tracks winningNumbers left
    int j = 0;
    // tracks how many matches found in loop
    int matchTally = 0;

    //  2.0 While i <= r1 AND j <= r2, repeat:
    while(i < playerTicket.length && j < winningNumbers.length) {

        // 2.1 If a1[i] < a2[j]:
        if(playerTicket[i] < winningNumbers[j]) {

            // 2.1.1 Increment i
            i++;

            // 2.2 If a1[i] > a2[j]:
        } else if(playerTicket[i] > winningNumbers[j]) {

            // 2.2.1 Increment j
            j++;

            // 2.3 If a1[i] == a2[j]
        } else {

            // 2.3.1 Increment matchTally
            matchTally++;

            // playerMatchString for checking individual ticket

            // update playerMatchString with matching array value
            playerMatchString += "[";

            // formatting: if value is less than 10, pad with leading zero
            if(winningNumbers[i] < 10) {
                playerMatchString += "0";
            }

            // complete the rest of the string
            playerMatchString += playerTicket[i] + "]";

            // increment i
            i++;
        }
    }

    // 3.0 Terminate
    return matchTally;
}
\end{javacode}
\end{listing}

\newpage
\subsubsection{``Merge search'' analysis}

Analysis of the ``merge search'' algorithm's time complexity involves counting the number of comparisons made during the operation. Let $n_1 = \mbox{right}1 - \mbox{left}1 + 1$ be the length of $a1[\mbox{left}1...\mbox{right}1]$, and let $n_2 = \mbox{right}2 - \mbox{left}2 + 1$ be the length of $a2[\mbox{left}2...\mbox{right}2]$. Let $n = n_1 + n_2$ be the total number of compared components \citep[p. 48]{Watt2001}.

Assuming that step 2 is implemented as a single comparison, the loop is repeated at most, $n - 1$ times \citep[p. 48 - 49]{Watt2001}. Therefore the time complexity is $O(n)$. Space complexity is of $O(1)$ since no copies are made during the operation.

\subsubsection{``Merge search'' console output}

The console output below displays the total number of winners in each winner class.
\\
\begin{consolecode}
***********************
**** MERGE METHOD *****
***********************

1st class winners: 0
2nd class winners: 0
3rd class winners: 4
4th class winners: 26
\end{consolecode}

\noindent
The console output below displays results for individual players.
\\
\begin{consolecode}
** MERGE TICKET CHECKING **

Player 0005 did not win. Thanks for playing lotto. 
Better luck next time!

Player 0500 is a 4th class winner!
Your winning numbers are: [11][33][45]

Player 0564 did not win. Thanks for playing lotto. 
Better luck next time!

Player 0897 did not win. Thanks for playing lotto. 
Better luck next time!
\end{consolecode}

\noindent
The console output below displays the result after a user inputs their player number.
\\
\begin{consolecode}
************************
****** USER INPUT ******
************************

ENTER YOUR PLAYER NUMBER TO CHECK IF YOU HAVE A WINNING TICKET:
500
** MERGE METHOD TICKET CHECK **
Player 0500 is a 4th class winner!
Your winning numbers are: [11][33][45]
\end{consolecode}

\newpage
\subsubsection{Binary search vs. ``merge search'' comparison}

\begin{itemize}
\item Binary search
	\begin{itemize}
	\item Time complexity: $O(\log n)$
	\item Space complexity: $O(1)$
	\end{itemize}
\item ``Merge search''
	\begin{itemize}
	\item Time complexity: $O(n)$
	\item Space complexity: $O(1)$
	\end{itemize}
\end{itemize}