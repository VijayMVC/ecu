\newpage
\section{Binary search trees}

The binary search tree data structure is demonstrated through a binary search tree of random integers. The following existing classes have been supplied, which construct the binary search tree:

\begin{itemize}
\item \mintinline{bash}{Tree}: Contains instance variable \mintinline{bash}{root} and defines logic to traverse nodes of a binary tree
\item \mintinline{bash}{TreeNode}: Contains instance variables for data, pointers for left/right of node and defines logic to insert a new node into the binary tree
\item \mintinline{bash}{TreeTest}: The executable class, contains \mintinline{bash}{main()} method
\end{itemize}

Additional methods have been added to the existing code of \mintinline{bash}{Tree} class to print all leaf nodes of a tree, print all non-leaf nodes of a tree and to determine and print the height of a tree.

These new methods have been tested with both the existing array of integers, which will be referred to as \mintinline{bash}{tree}, and with a more complex and larger array of integers, which will be referred to as \mintinline{bash}{bigTree}.

\begin{itemize}
\item \mintinline{bash}{tree} array: $\{49, 76, 67, 29, 75, 18, 4, 83, 87, 40\}$
	\begin{itemize}
	\item Figure 1 models the insertion of these integers into a binary search tree
	\end{itemize}
\item \mintinline{bash}{bigTree} array: $\{49, 76, 67, 29, 75, 18, 4, 83, 87, 40, 80, 46, 42, 43, 45, 41\}$
	\begin{itemize}
	\item Figure 2 models the insertion of these integers into a binary search tree
	\end{itemize}
\end{itemize}

\newpage
\begin{figure}[H]
\centering
\caption{\mintinline{bash}{tree} binary search tree}
\begin{tikzpicture}[very thick,level/.style={sibling distance=70mm/#1}]
% root
\node [vertex] (r){$49$}
	% left subtree
	child {
		node [vertex] (a) {$29$}
		child {
			node [vertex] {$18$}
			child {
				node [vertex] {$4$}
      		}
      		child {
      			node [vertex] {}
      		}
    	}
		child {
			node [vertex] {$40$}
    	}
	}
	% right subtree
	child {
		node [vertex] {$76$}
		child {
			node [vertex] {$67$}
			child {
				node [vertex] {}
			}
			child {
				node [vertex] {$75$}
			}
		}
		child {
			node [vertex] {$83$}
			child {
				node [vertex] {}
			}
			child {
				node [vertex] {$87$}
			}
		}
	};
\end{tikzpicture}
\end{figure}
\begin{figure}[H]
\centering
\caption{\mintinline{bash}{bigTree} binary search tree}
\begin{tikzpicture}[very thick,level/.style={sibling distance=70mm/#1}]
% root
\node [vertex] (r){$49$}
	% left subtree
	child {
		node [vertex] {$29$}
		child {
			node [vertex] {$18$}
			child {
				node [vertex] {$4$}
			}
			child {
				node [vertex] {}
			}
		}
		child {
			node [vertex] {$40$}
			child {
				node [vertex] {}
			}
			child {
				node [vertex] {$46$}
				child {
					node [vertex] {$42$}
					child {
						node [vertex] {$41$}
					}
					child {
						node [vertex] {$43$}
						child {
							node [vertex] {}
						}
						child {
							node [vertex] {$45$}
						}
					}
				}
				child {
					node [vertex] {}
				}
			}
		}
	}
	% right subtree
	child {
		node [vertex] {$76$}
		child {
			node [vertex] {$67$}
			child {
				node [vertex] {}
			}
			child {
				node [vertex] {$75$}
			}
		}
		child {
			node [vertex] {$83$}
			child {
				node [vertex] {$80$}
			}
			child {
				node [vertex] {$87$}
			}
		}
	};
\end{tikzpicture}
\end{figure}

\newpage
\subsection{Binary search tree traversal}

The traversal methods of pre-order, in-order and post-order were supplied as existing code, and therefore only the console output will be shown.

\subsubsection{Traversal console output}

The following console output demonstrates the insertion of values, pre-order, in-order and post-order traversals of \mintinline{bash}{tree}.
\\
\begin{consolecode}
Inserting the following values to tree: 
49 76 67 29 75 18 4 83 87 40 

Pre-order traversal of tree:
49 29 18 4 40 76 67 75 83 87 

In-order traversal of tree:
4 18 29 40 49 67 75 76 83 87 

Post-order traversal of tree:
4 18 40 29 75 67 87 83 76 49 
\end{consolecode}

\noindent
The following console output demonstrates the insertion of values, pre-order, in-order and post-order traversals of \mintinline{bash}{bigTree}.
\\
\begin{consolecode}
Inserting the following values to bigTree: 
49 76 67 29 75 18 4 83 87 40 80 46 42 43 45 41 

Pre-order traversal of bigTree:
49 29 18 4 40 46 42 41 43 45 76 67 75 83 80 87 

In-order traversal of bigTree:
4 18 29 40 41 42 43 45 46 49 67 75 76 80 83 87 

Post-order traversal of bigTree:
4 18 41 45 43 42 46 40 29 75 67 80 87 83 76 49
\end{consolecode}

\newpage
\subsection{Print leaf nodes only}

A leaf node of a binary tree is a node which does not reference a left node, and does not reference a right node. The following \emph{recursive} method has been created to print all the leaf nodes in a binary search tree. The method utilizes in-order traversal.

\subsubsection{Print leaf nodes only Java method}

\begin{listing}[H]
\caption{Print leaf nodes only method}
\begin{javacode}
// public method, initiates recursive printLeafHelper()
public synchronized void printLeafOnly() {
    printLeafHelper(root);
}

// private recursive method which traverses the tree
private synchronized void printLeafHelper(TreeNode node) {
    if(node == null) {
        return;
    }

    printLeafHelper(node.left);

    // only print data if left and right are null
    if(node.left == null && node.right == null) {
        System.out.print(node.data + " ");
    }

    printLeafHelper(node.right);
}
\end{javacode}
\end{listing}

\subsubsection{Print leaf nodes only console output}

The following console output demonstrates the execution of Java code 4.1 with the original \mintinline{bash}{tree} binary search tree.
\\
\begin{consolecode}
Print leaf nodes only for tree:
4 40 75 87 
\end{consolecode}

\noindent
The following console output demonstrates the execution of Java code 4.1 with the \mintinline{bash}{bigTree} binary search tree.
\\
\begin{consolecode}
Print leaf nodes only for bigTree:
4 41 45 75 80 87 
\end{consolecode}

\newpage
\subsection{Print non-leaf nodes only}

A non-leaf node is the inverse of a leaf node in a binary tree. A non-leaf node references either a left, or right node. The following \emph{recursive} method has been created to print all non-leaf nodes within a binary search tree. The method utilizes in-order traversal.

\subsubsection{Print non-leaf nodes only Java method}

\begin{listing}[H]
\caption{Print non-leaf nodes only method}
\begin{javacode}
// public method, initiates recursive printNonLeafHelper()
public synchronized void printNonLeafOnly() {
    printNonLeafHelper(root);
}

// private recursive method which traverses the tree
private synchronized void printNonLeafHelper(TreeNode node) {
    if(node == null) {
        return;
    }

    printNonLeafHelper(node.left);

    // only print data if left or right != null
    if(node.left != null || node.right != null) {
        System.out.print(node.data + " ");
    }

    printNonLeafHelper(node.right);
}
\end{javacode}
\end{listing}

\subsubsection{Print non-leaf nodes only console output}

The following console output demonstrates the execution of Java code 4.2 with the original \mintinline{bash}{tree} binary search tree.
\\
\begin{consolecode}
Print non-leaf nodes only for tree:
18 29 49 67 76 83 
\end{consolecode}

\noindent
The following console output demonstrates the execution of Java code 4.2 with the \mintinline{bash}{bigTree} binary search tree.
\\
\begin{consolecode}
Print non-leaf nodes only for bigTree:
18 29 40 42 43 46 49 67 76 83 
\end{consolecode}

\newpage
\subsubsection{Print height of tree}

The height (also known as depth) of a binary search tree is the depth of the deepest node in a binary search tree. The following \emph{recursive} method is an adaptation of a Java method by \citet{Flaschen2014}. The method has been modified to follow convention that if a binary tree is empty, the method will return $-1$ \citep[p. 239]{Watt2001}.

\subsubsection{Print height of tree Java method}

\begin{listing}[H]
\caption{Print height of tree method}
\begin{javacode}
// public method, initiates recursive getHeightHelper()
public synchronized int getHeight() {
    return getHeightHelper(root);
}

// adaptation of Flaschen's method (2014)
private synchronized int getHeightHelper(TreeNode node) {
    if(node == null) {
        return -1;
    }

    if(node.left == null && node.right == null) {
        return 0;
    }

    return Math.max(getHeightHelper(node.left), getHeightHelper(node.right)) + 1;
}
\end{javacode}
\end{listing}

\subsubsection{Print height of tree console output}

The following console output demonstrates the execution of Java code 4.3 with the \mintinline{bash}{tree} binary search tree.
\\
\begin{consolecode}
Print height of tree:
3
\end{consolecode}

\noindent
The following console output demonstrates the execution of Java code 4.3 with the \mintinline{bash}{bigTree} binary search tree.
\\
\begin{consolecode}
Print height of bigTree:
6
\end{consolecode}

\noindent
The following console output demonstrates the execution of Java code 4.3 where a binary search tree is either empty, or only contains the root node.
\\
\begin{consolecode}
Print height of emptyTree:
-1

Print height of oneNodeTree:
0
\end{consolecode}