\section{Security analysis}

\subsection{Operating system}

\subsubsection{Windows Vista Business x86 SP0}

The VM image provided includes Windows Vista Business 32 bit (x86) Service Pack 0 (SP0) as its operating system. Windows Vista is considered to be an older operating system with extended support due to end on 11 April, 2017 \citep{Microsoft2014}. Service Pack 0 indicates that no service packs have been installed. In addition, SP0 support has ended on 13 April, 2010. End of support means that the operating system will ``no longer receive security updates that can help protect your PC from harmful viruses, spyware, and other malicious software that can steal your personal information'' \citep{Microsoft2015}.

Analysis of the VM image has shown that no service packs or security updates from Microsoft have been installed, leaving the operating system vulnerable to various exploits or attacks. For example, CVE-2008-0951 describes an ``Execute Code'' vulnerability in the operating system's Auto-run function, which allows arbitrary code on removable media to be executed when inserting either a CD-ROM or USB device \citep{SecurityFocus2008}. This vulnerability would allow a user-assisted attacker to insert malicious code on a CD-ROM or USB hard drive, and insertion of the CD-ROM or USB hard drive would result in the execution of the malicious code.

Additionally, CVE-2007-5133 describes a ``Denial of Service Overflow'' vulnerability in Microsoft Windows Explorer. Windows Explorer's inability to handle malformed PNG image files can be exploited, causing the CPU to waste cycles which results in an unresponsive system \citep{SecurityFocus2007a}. For example, a user-assisted attacker may send a victim a website link with an embedded malformed PNG image file, causing the computer to lock up when viewed by the victim.

Furthermore, CVE-2007-1658 describes a vulnerability in Windows Mail, an email application packaged with the operating system. The vulnerability allows a user-assisted attacker to send an email containing a ``maliciously crafted link'' to execute local files \citep{SecurityFocus2007b}.

As seen in the examples above, withholding the installation of operating system service pack and/or security updates leave the system open to various vulnerabilities. Installing regular operating system updates is a step towards a secure system, ensuring that the latest known vulnerabilities are patched, and should be performed regularly as best practice. Blue Ink should also note the end of extended support date. In approximately two years from the time of writing this report, no further security updates to the operating system will be supplied by the vendor. This will be a point to consider if any sensitive information is planned to be stored on these systems after the ``end of support'' date.

\subsubsection{Internet Explorer 7}

Internet Explorer 7 (IE 7) is the default browser included with Microsoft Vista. IE 7 is considered an older browser and will no longer receive support or security updates after 12 January, 2016 \citep{Capriotti2014}. Furthermore, since no service packs or security updates have been applied to the operating system itself, no security patches have also been applied to this browser, rendering the browser vulnerable to various attacks and zero-day exploits.

For example, CVE-2015-0067 describes a ``Remote Memory Corruption'' vulnerability for Internet Explorer versions 6 through 9. A remote user-assisted attacker may exploit this vulnerability by sending the victim a link to view a ``specially crafted'' web page which can then execute malicious code \citep{SecurityFocus2015}.

Additionally, CVE-2013-3918 describes a vulnerability in an ActiveX Class which allows an attacker to remotely execute code on a victim's computer through a malicious website \citep{Ozkan2013, Microsoft2013}. \citet{Chen2013} assert that this particular vulnerability was exploited in a ``watering hole'' attack on a US based website. Anyone that visited the website was subject to a ``drive-by download'' enabled by the vulnerability, in order to propagate malware. \citet{Moran2013} have also found that the payload attacks the memory directly without writing to disk, and ``generally cannot be detected by traditional anti-malware tools'' \citep{Wilson2013}.

There are many more vulnerabilities found for this version of Internet Explorer, and the list will continue to grow as they are found. The two examples above indicate the importance of installing security software updates regularly to ensure that known vulnerabilities and exploits are patched. As with the end of support date for Windows Vista, the support of Internet Explorer 7 will only continue for less than a year, from the time of writing this report. Consideration to update to a modern, supported browser is suggested to ensure the security of the business' data.

\subsubsection{Windows Defender}

Windows Defender is an anti-malware software included with Windows Vista. 

\subsection{Applications}

\subsubsection{Anti-virus}

\subsubsection{Adobe Reader 6.0.1}

\subsubsection{OpenOffice 1.1.5}

\subsubsection{mIRC 6.0}

\subsection{User practices}

Include user account "green" is admin. Why? Should not have admin rights.