\section{Conclusion}

In conclusion, a number of security issues have been identified in the operating system, installed applications and user practices during the analysis of Blue Ink's VM image. The current operating system, Windows Vista Business does not have any service packs or security updates installed. Known vulnerabilities for the current version of the operating system and bundled browser, Internet Explorer 7, have not been patched due to the lack of installed updates. The end of support date for the operating system and Internet browser must also be considered as no further security updates will be provided by the vendor from that date onward.

In addition, bundled anti-malware software, Windows Defender, does not have up-to-date virus definitions, rendering the anti-malware software ineffective to current threats. Windows Defender was also tested with \href{http://www.eicar.org/85-0-Download.html}{www.eicar.org} pseudo-viruses, and failed to detect them.

A fake anti-virus resides on the computer. The origin of which is unknown, but potentially could have been installed by malware, luring the user with a false sense of protection. Furthermore, installed software such as Adobe Reader and OpenOffice have not been updated to their latest version. These legacy versions have been identified with vulnerabilities and leave the system vulnerable to attacks.

Access to IRC clients must also be re-considered. In addition to software vulnerabilities, IRC clients share similar risks of malware propagation as email. Attackers can easily share malicious links or files through IRC.

User practices have also been found not to follow security best practices. The computer boots into the operating system without asking for an account password, and logs into an administrator account by default. The screen-saver is also not protected by password. These practices leave the computer vulnerable to local attackers.

A plain-text file has been found to store a password manager master password. If the computer were to be compromised by an attacker, the master password would be easily accessed, and therefore all passwords stored and encrypted by the password manager would also be compromised.

Finally, standard users are being provided with administrator privileges. If such an account were to be compromised, the attacker could potentially inherit the user's elevated access. The disabling of UAC also lead to muted notifications if system changes were made by malware.

%In order to mitigate these security issues, this report suggests that frequent operating system and application updates must be installed along with updates to Windows Defender virus definitions. The fake anti-virus must be removed and replaced, or depend solely on an updated Windows Defender. mIRC should be removed from the computer to avoid the risk of exposure to malicious links and files and passwords of any kind should not be stored in plain-text files. Normal users should not be provided with administrator rights and UAC should be re-enabled.

