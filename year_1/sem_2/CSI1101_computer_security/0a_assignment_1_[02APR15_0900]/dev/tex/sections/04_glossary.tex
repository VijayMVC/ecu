\section{Glossary}

\begin{itemize}
\item ActiveX Control: A small program which acts as an add-on or extension for Internet Explorer
\item Arbitrary code execution: To run commands that are not part of the vulnerable application code
\item Auto-run: A feature of Microsoft Windows operating systems which automatically open files or execute actions when removable media has been mounted on the system
\item Buffer Overflow: Occurs when a program attempts to utilise more data in its buffer than it can hold, or attempts to place data in an area of memory  past its allocated buffer
	\begin{itemize}
	\item Corruption of data, denial of service conditions or arbitrary code may be executed if this occurs	
	\end{itemize}
\item Denial of Service: An interruption in access, typically with malicious intent
\item Firewall: Software or hardware that protects a Local Area Network from unwanted network traffic, acts as a filter
\item Heap memory corruption: Similar to buffer overflow, can be caused by the program utilising memory which has not been assigned to the program
	\begin{itemize}
	\item Similar consequences to buffer overflow
	\end{itemize}
\item HTML: Hypertext Markup Language, a set of markup tags used to create a web page
\item IRC: Internet Relay Chat, a network protocol that allows communication through text or chat
	\begin{itemize}
	\item Also facilitates link and file-sharing
	\end{itemize}
\item Java: A programming language
\item Java Applet: A small embedded application written in Java programming language
\item LAN: Local Area Network, ie. Network within own walls
\item PDF: Portable Document Format
\item Sandbox: A security measure to partition the access of applications to the system
\item SP0: Service Pack 0 (zero), no service pack updates installed for the base operating system
\item User assisted attack: An attack where the user must take action before the attack can begin
	\begin{itemize}
	\item An attacker may use social engineering to convince victim to take required action
	\end{itemize}
\item WAN: Wide Area Network, ie. the Internet
\item Watering hole attack: Target is usually a group, or organisation and attacks website/s which is known for members of the group to visit often
	\begin{itemize}
	\item The website/s are then infected with malware, which then propagates to members of the target group visiting the website, and then spread it to other members of the target group 
	\end{itemize}
\item x86: 32-bit operating system
\end{itemize}