\section{Literature review}

\citet{Fallows2005} compared American male and female Internet users and found that men were more likely to use the Internet in general for information gathering purposes, while women were more likely to use the Internet for social applications to maintain current relationships. According to the literature in this review, this statement also translates to the use of SNS such as Facebook.

\subsection{Social gender role theory and gender motivations}

\citetapos{Eagly1987} theory of social gender roles introduces a framework to explain differences in gender, regardless of being online or offline, which underpins many cyber-psychology studies \citep{Hum2011, Tifferet2014, Kimbrough2013}. This theory characterizes men as ``\emph{agentic} providers'' and women as ``\emph{communal} caregivers'' \citep[p. 1830]{Hum2011}. In other words, according to this theory, men tend to develop traits which lend to task-based activities, while women tend to develop traits which lend to social interactions. The theory of social gender roles is one of many frameworks used to explain the differences between gender in SNS and Internet use, which is evident in \citetapos{Fallows2005} generalisation that men use the Internet for information gathering purposes, and women use the Internet for social activities.

\subsection{Male gender role and motivations}

Widely cited research by \citet{Raacke2008} was among the first to examine the impact of SNS on college students and observed that men, compared to women, were more likely to use SNS to find out about events, indicating that men spend more time than women performing information-gathering activities on SNS. In support of this view, \citet[p. 2]{Choi2014} hypothesise that men have higher positive attitudes towards SNS advertising than women, as men are more likely to perceive such advertising as ``useful information'' due to their ``information-oriented motivation''. The presented evidence supports \citetapos{Eagly1987} social gender theory. On the other hand, \citet{Park2009} argues by claiming that women were more likely to use Facebook Groups for obtaining information. However, this study only examines the use of Facebook \emph{Groups} and not Facebook as a whole. 

In comparison to women using SNS as a medium to maintain existing relationships, men have been found to use SNS as a tool for creating new relationships and expanding their networks \citep{Mazman2011}. This view is supported by findings in research by \citet{Muscanell2012}, \citet{Raacke2008} and \citet{Haferkamp2012}, which illustrates that men are more likely to use SNS for dating purposes than women. These findings reiterate a difference in motivation of SNS use between genders. 

\subsection{Female gender role and motivations}

In contrast, \citet{Mazman2011} and \citet{Muscanell2012} both assert that women are more likely to seek out old friends on the network, and are more likely to utilise SNS communication tools to maintain existing relationships. To support this statement, \citet{Joiner2014} provides evidence that women are more likely to demonstrate higher emotional support in response to a friend's negative Facebook status update. Women are also twice as likely to respond publicly to a negative status update when compared to men \citep[p. 167]{Joiner2014}. These statements lend to the hypothesis that women use SNS as a tool for relationship maintenance more than men, which align with \citetapos{Eagly1987} theory of social gender roles.

As the female gender role is more concerned with socialising than men, it could be assumed that women would have more friends within their SNS network. \citet{Raacke2008} and \citet{Fogel2009} found the contrary, demonstrating that men had more friends than women, which could support the theory that men use SNS to expand their networks more than women. Nevertheless, more recent research by \citet{McAndrew2012} found that women have more friends than men. The contradiction in findings could be due to the difference in time frames in which these studies occurred, and as \citet{Fallows2005} suggest, women have since caught up to men in Internet connectedness. However, according to \citet[p. 389]{Tifferet2014}, there are many studies that have conflicting conclusions relating to gender and network size, which warrants further investigation.

\subsection{Limitations and challenges}

It is worth noting that studies by \citet{Raacke2008}, \citet{Muscanell2012} and \citet{Joiner2014} were limited to participants from a single American college comprised of first-year undergraduate students who provided self-reported estimates. Studies based on observed data have the potential to increase research reliability and enhance conclusive results when compared to studies based on self-reported estimates.

Interestingly, \citet{Raacke2008} and \citet{Joiner2014} gathered data from respondents via paper questionnaires, as opposed to \citetapos{Mazman2011} and \citetapos{Muscanell2012} online questionnaire method. Online questionnaires, while providing convenience and ability to reach a far greater number of participants compared to paper questionnaires, have the potential to skew results towards users who may spend more time online, possibly use SNS more, and have higher competency in SNS use, compared to those who spend less time online \citep[p. 280]{Hargittai2007}.

Competency in SNS use, classed within ``Computer Mediated Communication (CMC) Competency'' measures by \citet[p. 579]{Ross2009}, was a variable largely ignored in most of the research within the scope of this review, which could ``influence how much people use social networking sites'' \citep[p. 898]{Kimbrough2013}. Without the measurement of CMC competency, it is only assumed that all participants of such studies are equally skilled in the use and application of SNS, which certainly may not be the case, as \citet{Ross2009} suggests.

%\citetapos{Choi2014} research provides a unique perspective in the role of gender in Facebook use, comparing the relationship of self-presentation on brand-related word-of-mouth and gender's moderating effects. According to \citet{Choi2014}, only a small amount of research has been completed in the area of brand-related word-of-mouth on SNS. However, this study was limited to participants from the same country, providing self-reported estimates. The study was also limited to respondents within the ages of ninteeen to thirty-nine, as \citet[p. 3]{Choi2014} claims that age bracket represented the ``primary Facebook user population''.

\citetapos{Haferkamp2012} study was based on randomly selected users from \citet{StudiVZ2014}, a German SNS for students. At the time of research in 2010, the majority of StudiVZ profiles were public and ``used without privacy settings'' \citep[p. 92]{Haferkamp2012}, which allowed the study of observed data from participant profiles, together with self-reported online questionnaire results. Although \citetapos{Haferkamp2012} research utilized observed data from respondents from another country while using a completely unique SNS, the results were consistent with most of the literature in this review.