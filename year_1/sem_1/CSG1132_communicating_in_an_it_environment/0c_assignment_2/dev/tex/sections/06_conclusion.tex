\section{Conclusion}

The aim of this research paper is to explore the relationships between gender and Facebook use. Observations were collected from 48 fictional Facebook users who participated in a survey and the following thesis statements were tested by comparing gender and the number of Facebook friends, the number of close friends, the Sociability score and reported Facebook hours. 

Out of the total 48 participants, only 3 were women. The data collected from the participants were identified as non-normally distributed, and therefore non-parametric tests were selected. 

\subsection{Thesis statement 1: Gender is related to the size of a user's Facebook network}

This research paper's literature review discovered that there were conflicting views on the topic of gender and network size. Gender was compared with Facebook Friends to find any supporting evidence that gender is related to the size of a user's Facebook network.

Spearman's correlation coefficient test and Wilcox rank sum test both indicate that there is no correlation between Gender and Facebook Friends, however due to the small number of female participation, no conclusions can be made with the results.

Secondary variables were also explored, testing Close Friends and Sociability with Gender, which may have provided further evidence towards the thesis statement. Both non-parametric tests indicate that there is no correlation between Gender and Close Friends or Sociability. With such a small number of female participants, no conclusions can be made with the results.

\subsection{Thesis statement 2: Gender is related to the amount of time a user spends on Facebook}

The literature review suggested that women are attracted to the social aspect of Facebook and SNS more than men. This research paper aimed to explore that statement by comparing Gender with reported Facebook Hours.

Again, both non-parametric tests found that there is no correlation between Gender and Facebook Hours. However, there is not enough evidence from female participants for conclusions to be made with the results.

\subsection{Limitations}

Due to the small amount of female representation in the sample set, no conclusions from the test results could be made, and posed a recurring limitation in this research paper. Further research is recommended with a sample set which includes an equal balance between male and female participants. Actual Facebook friend counts rather than self-reported estimates would also further enhance any future study results.