\section{Introduction}

%Facebook is one of today's leading Social Networking Sites (SNS). The company reports that as of June 2014, their network serves 1.32 billion users per month \citep{Facebook2014}, while independent studies have shown that SNS users were more likely to be using Facebook \citep{Hampton2011, Raacke2008, Hargittai2007}. As Facebook expanded registration to users outside educational and professional institutions in September 2006 \citep{Facebook2014}, males and females alike were quick to adopt the technology at a fast paced rate \citep{Mazman2011}. This adoption rate has triggered a multitude of scientific research ``from widely different fields of inquiry'', attempting to explain the phenomenon of Facebook \citep[p. 983]{Caers2013}.

%Facebook is a Social Networking Site (SNS) that allows its users to communicate with each other through various public and private methods. Users are able to customise their profile by including a profile picture and upload personal photographs. One of the main functions of Facebook is the ability to build friendship networks, where users can send friendship requests to one another which can be accepted or rejected. After a friendship connection has been made, two users are then able to communicate with each other through various means. 

This research paper explores the role of gender and its effects relating to personal Facebook use. Facebook is one of today's leading Social Networking Sites (SNS) and the company reports that as of June 2014, their network serves 1.32 billion users per month \citep{Facebook2014}. Independent studies have also shown that SNS users were more likely to be using Facebook than any other SNS site \citep{Hampton2011, Raacke2008, Hargittai2007}. As Facebook expanded registration to users outside educational and professional institutions in September 2006 \citep{Facebook2014}, users were quick to adopt the technology at a fast paced rate \citep{Mazman2011}. This adoption rate has triggered a multitude of scientific research ``from widely different fields of inquiry'', attempting to explain the phenomenon of Facebook \citep[p. 983]{Caers2013}. 

Differences in gender is a facet of SNS use that has been investigated by many researchers \citep{Fallows2005, Haferkamp2012, Hargittai2007, Joiner2014, Kimbrough2013, Mathiyalakan2014, Mazman2011}. Men have been generally regarded as earlier adopters of technology compared to women, and the adoption of SNS such as Facebook has been no different. \citetapos{Pitkow1994} study showed that during the mid 1990's, 95\% of Internet users were men, while research by \citet{Fogel2009} demonstrates that men are also earlier adopters of SNS, finding that more men had established accounts before women. However, the trend has shifted, with recent reports indicating that women now represent the majority of SNS users compared to men \citep{Duggan2013, Hampton2011}.

\citetapos{Eagly1987} theory of social gender roles introduces a framework to explain differences in gender, which underpins many cyber-psychology studies \citep{Hum2011, Kimbrough2013, Tifferet2014}. This theory characterizes men as ``\emph{agentic} providers'' and women as ``\emph{communal} caregivers'' \citep[p. 1830]{Hum2011}. In other words, according to this theory, men tend to develop traits which lend to task based activites, while women tend to develop traits which lend to social interactions.