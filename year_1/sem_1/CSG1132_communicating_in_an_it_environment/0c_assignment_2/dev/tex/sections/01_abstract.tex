\section{Abstract}

This research paper explores the role of gender and its effects relating to personal Facebook use. Facebook is one of today's leading Social Networking Sites (SNS) and the company reports that as of June 2014, their network serves 1.32 billion users per month \citep{Facebook2014}.

Through this paper's literature review, an inconsistent view of gender and its relation to Social Networking Site (SNS) network size was identified. The first thesis statement that this paper will investigate is ``Gender is related to a user's Facebook network size''. The literature reviewed also indicate that women are attracted to the social aspect of Facebook more than men. Since the majority of activities on SNS relates to socialising, this research paper suggests that gender is related to the amount of time a user spends on Facebook. To investigate this view, the second thesis statement to be explored is ``Gender is related to the amount of time a user spends on Facebook''.

The current sample set consists of 48 fictional participants, 3 female, 45 male (M = 0.938, SD = 0.245, gender coded as 0 = female, 1 = male) between the ages of 17 to 29 (M = 20.6, SD = 3.207) from a university in Perth, Western Australia, who responded to a survey which included questions about their personal Facebook use, demography and personality.

The following variables were compared with Gender in response to the thesis statements: Facebook Friends, Close Friends, Sociability and Facebook Hours. The selected variables were all identified with exhibiting non-normal distributions and non-parametric tests, Spearman's correlation and Wilcox rank sum, were chosen to test correlation and hypotheses.

Results of the tests found no correlation between gender and the chosen variables. However, with such a small number of female observations in the study, there is not enough evidence to provide any conclusions from the results.