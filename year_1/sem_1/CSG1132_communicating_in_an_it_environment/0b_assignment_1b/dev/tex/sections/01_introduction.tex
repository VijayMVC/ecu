\section{Introduction}

Facebook is one of today's leading Social Networking Sites (SNS). The company reports that as of June 2014, their network serves 1.32 billion users per month \citep{Facebook2014}, while independent studies have shown that SNS users were more likely to be using Facebook \citep{Hampton2011, Raacke2008, Hargittai2007}. As Facebook expanded registration to users outside educational and professional institutions in September 2006 \citep{Facebook2014}, males and females alike were quick to adopt the technology at a fast paced rate \citep{Mazman2011}. This adoption rate has triggered a multitude of scientific research ``from widely different fields of inquiry'', attempting to explain the phenomenon of Facebook \citep[p. 983]{Caers2013}. 

This literature review focuses on research which examine the demographic differences relating to SNS use, with particular attention to the role of gender. Men are generally regarded as earlier adopters of technology compared to women. This is evident in findings by \citet{Pitkow1994}, where 95\% of Internet users were men, while \citet[p. 896]{Kimbrough2013} declares that during the first half of the 1990's, the Internet ``was mostly regarded as a technological boy's toy''. A decade later, \citet{Fallows2005} asserts that there are just as many females as there are males online. Research by \citet{Fogel2009} provides supporting evidence that men are also earlier adopters of SNS, finding that more men had established SNS accounts before women. However, the trend has shifted, with recent reports indicating that women now represent the majority of SNS users compared to men \citep{Duggan2013, Hampton2011}.

As the Internet user gender gap disappears, it is more important than ever to understand the differences between genders and its effects relating to personal SNS use, so that current social network sites and social network sites of the future are able to service both men and women equally. This literature review will explore those differences as expressed in the body of current research literature, by analysing the following topics:
\begin{itemize}
\item Motivational differences between genders
\item Self-presentation differences between genders
\item Difference in SNS privacy concerns between genders
\end{itemize}