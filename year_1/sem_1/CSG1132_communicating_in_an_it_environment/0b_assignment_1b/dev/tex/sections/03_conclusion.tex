\section{Conclusion}

\citet[p. 897]{Kimbrough2013} succinctly pointed out that while users have the ability to choose to behave in any way they wish online, men and women still conform to behaviour that is consistent with ``social role expectations'' from the offline world. This literature review has explored the differences between genders in the use of SNS, and has identified key differences in user motivation. Studies have shown that men are more likely to perform information-based activities on SNS than women, which align with the social gender role framework introduced by \citet{Eagly1987}. Men are also more open to expanding their networks and use SNS as a tool to create new relationships, and more frequently use SNS as a dating platform than women. Women on the other hand, have been found to use SNS to maintain current relationships, and are generally more predisposed than men to provide emotional support to their friends. %Women are attracted to the socialising aspect of SNS, and recent studies have shown that women have a larger social network than men. These key differences are also consistent with social gender theory.

As indicated by \citet{Tifferet2014}, gender and its relation to SNS network size is found to be a topic of debate. Studies have reported conflicting results and requires the attention of further investigation. The exploration of this domain will be the basis of subsequent analysis from this researcher.

%This literature review has identified some motivational differences in the use of SNS and compared them to social gender role expectations with the aim of better understanding the differences between genders in SNS use.