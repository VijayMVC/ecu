\documentclass[a4paper, 12pt]{article}
% \usepackage[citestyle=apa, backend=biber, hyperref=true]{biblatex}
\usepackage[backend=biber]{biblatex}
\bibliography{../../../../../../../../bib/library.bib}

\title{CSG1132 Assignment 1A: Learning Summary}
\author{Martin Ponce}
\date{\today}

\begin{document}

\maketitle

\section{Concept mapping summary}

The concept map was developed in 1972 by researchers at Cornell Univesity and is used as a tool to visually organise and present knowledge, usually in reference to a focus question or topic\citep*{Novak2006, Hilbert2009}. \citet[pp. 267]{Hilbert2009} explains that concept mapping is ``based on Ausebel's assimilation theory of cognitive learning''. A concept map consists of a collection of concepts that relate to a particular subject or problem domain, and are written inside bubbles or `nodes' which are then linked together with lines to display their associations with one another. Unlike mind maps, these lines are labelled to express the relationships between these linked concepts, otherwise referred to as ``path labelling'' by\citet[pp. 790]{Rodriguez-Priego2013}. 

Concept maps are structured in a hierarchical sense, where the most general concepts are displayed at the top of the map, and more specific, less inclusive concepts are drawn towards the bottom of the map\citep{Novak2006}. Another characteristic of a concept map is its non-linearity. Concept maps have the ability to `cross-link' and display relationships between concepts in separate domains within the given topic. Such cross-links demonstrates that the researcher has developed a deeper understanding of the topic, has created new understanding through the research, and is able to express relationships and propositions between outlying concepts that may have not been immediately evident prior to research\citep{Novak2006}. This method is in contrast to a mind map where such flexibility or specificity is not required to be expressed, and concepts are only linked to their direct counterparts without the ability to label identified relationships explicitly.

According to\citet[pp.1]{Novak2006}, a concept map ``serves as a kind of \textit{template} or \textit{scaffold} to help organize knowledge and to structure it''. It is due to this structure that concept maps may be used to define the scope of the topic to be researched and assists in identifying knowledge gaps in domains where further research could be gathered\citep{Novak2006}. As stated previously, concept maps display information in an organised, non-linear manner which allows for scalablility so that new concepts can be logically connected to pre-existing nodes and assists in organising information as they are found during research\citep{Novak2006}. Concept maps facilitate in the organisation of information gathered from study or research from many sources in a coherent and flexible manner which makes it a valued tool for learning.

\section{Learner reflection}

As I enrolled for university as a first year, undergraduate student, one of my main concerns was academic writing. I understood that this was something I would be expected to do and was concerned about my ability to perform in this area. Concept mapping has helped me break down the overwhelming amount of information from peer-reviewed journal articles into smaller digestable pieces and make sense of the information I was taking in.

In hindsight, I would change my approach in obtaining material for my reading list. To begin with, I downloaded as many related journal articles as I could. The ability to create advanced search queries within Library One provided a great deal of articles and some databases also provided related suggestions which I constantly followed. My reading list became increasingly large and found it difficult to identify which documents I have read and annotated.After the Week 3 workshop, I stopped annotating the PDFs and wrote notes based on the template provided in the workshop. This helped a great deal in identifying which articles have been read, and will continue to do this in future research. 

After getting a grasp of the topic at hand, I developed a sense of which documents would become valuable for the assignment and scanned the abstracts to determine if the article is relevant to my research. This allowed me to cut down on the noise building up in my reading list.

As I began to develop the concept map, I found it difficult to organise the information in any kind of hierarchy and started my way from the bottom, with the most specific concepts. This resulted in concept maps that seemed more like mind maps. I found\citet{Novak2006}``The Theory Underlying Concept Maps and How To Construct and Use Them'' a valuable resource and used the methods outlined in the article. In particular, I found the `parking lot' method the most useful, dropping concepts into map as I was reading articles, and then decide their associations after I finished reading and developed a better understanding of the concept relationships.

Once a good structure and foundation for the concept map was established, it was easier to identify cross-links between different concepts. The difficult part was deciding which cross-links were most important, and deciding how to present these links with so many lines intersecting and overlapping each other. This resulted in many iterations of the map, attempting to arrange concepts in a heirarchical sense while accomodating for these cross-links.

Completing the second task, writing thesis statements based on articles, provided a different view and helped identify even more cross-links from the articles. Completing this exercise also highighted the most important cross-links and has motivated me to redesign the concept map once more to emphasise those links.

As a student, I believe concept mapping is a great tool and appreciate its value when used correctly for academic writing. It helps make sense of information that come from a variety of sources and assists in identifying/generating new ideas. I have started using the tool for other units, and find it particularly useful for systems analysis, where there are many moving parts and allows me to get a broad overview of the system before making decisions. I understand there are other industry accepted ways of modelling a system, but Cmap allows me to visualise a system far more quickly with great detail. I look forward to more opportunities to use concept mapping and truly appreciate being introduced to this wonderful tool for learning.




\bibliographystyle{apa}
\end{document}